\documentclass[
a5paper,
%paper=5.5in:8.5in,
]{scrbook} % Document font size and paper size

\title{A Christmas Carol}
\usepackage[namedChapters]{bindery}
\usepackage{dropcaps}
\usepackage{illustrations}

\usepackage{enumitem}
\setlist[itemize]{noitemsep} 

\setfontfamily\chapterfont{Menuetto.ttf}
\setdropcaps{acorninitials.ttf}


\DeclareTOCStyleEntry[beforeskip=.1cm,linefill=\TOCLineLeaderFill, dynnumwidth=true]{chapter}{chapter}

\DeclareTOCStyleEntry[beforeskip=.1cm,numwidth=-10pt,entrynumberformat=\hideentrynumber,linefill=\TOCLineLeaderFill]{tocline}{figure}

\hyphenation{Crat-chit  Eb-en-ez-er}

\begin{document}

\newcommand{\moderatelyhuge}{\fontsize{40}{50}\selectfont}



\frontmatter
\renewcommand*{\sectionmarkformat}{}



\KOMAoptions{headings=openright}
\includepdf[width=\basicwidth]{titlepage.jpg}
\thispagestyle{empty}

\KOMAoptions{headings=openany}

\chapter{Preface}

\lettrine[lines=4]{I}{} have endeavoured in this Ghostly little book to raise the Ghost of an Idea which shall not put my readers out of humour with themselves, with each other, with the season, or with me. May it haunt their house pleasantly, and no one wish to lay it.

~\\

Their faithful Friend and Servant,

\begin{flushright}
\textsc{C.D.}
\end{flushright}

\textit{December, 1843.}

\vfill
\begin{figure}[h]
\centering
\includegraphics[width=.5\textwidth]{gs007}
\caption[Tailpiece to Preface]{}
\end{figure}

\pagestyle{plain}

\tableofcontents
\vfill
\begin{figure}[h!]
\centering
\includegraphics[width=0.7\textwidth]{elfturkey}
\caption[Tailpiece to Table of Contents]{}
\end{figure}
%\clearpage

\KOMAoptions{headings=openleft}
\renewcommand{\listfigurename}{List of Illustrations}
\listoffigures
\vfill
\begin{figure}[h]
\centering
\includegraphics[width=.9\textwidth]{doublelf}
\caption[Tailpiece to List of Illustrations]{}
\end{figure}
\clearpage

\chapter{Characters}
\begin{itemize}
\item[\ding{100}] \textbf{Bob Cratchit}, clerk to Ebenezer Scrooge.
\item[\ding{101}] \textbf{Peter Cratchit}, a son of the preceding.
\item[\ding{102}] \textbf{Tim Cratchit (\enquote*{Tiny Tim})}, a cripple, youngest son of Bob Cratchit. 
%the enquote* forces the use of enquote's currently set INNER quotes. Since I'm using British style quotations, inner quotes are "", which is what I want.
\item[\ding{100}] \textbf{Mr Fezziwig}, a kind-hearted, jovial old merchant.
\item[\ding{101}] \textbf{Fred}, Scrooge's nephew.
\item[\ding{102}] \textbf{Ghost of Christmas Past}, a phantom showing things past.
\item[\ding{100}] \textbf{Ghost of Christmas Present}, a spirit of a kind, generous, and hearty nature.
\item[\ding{101}] \textbf{Ghost of Christmas Yet to Come}, an apparition showing the shadows of things which yet may happen.
\item[\ding{102}] \textbf{Ghost of Jacob Marley}, a spectre of Scrooge's former partner in business.
\item[\ding{100}] \textbf{Joe}, a marine-store dealer and receiver of stolen goods.
\item[\ding{101}] \textbf{Ebenezer Scrooge}, a grasping, covetous old man, the surviving partner of the firm of Scrooge and Marley.
\item[\ding{102}] \textbf{Mr Topper}, a bachelor.
\item[\ding{100}] \textbf{Dick Wilkins}, a fellow apprentice of Scrooge's.
\item[\ding{101}] \textbf{Belle}, a comely matron, an old sweetheart of Scrooge's.
\item[\ding{102}] \textbf{Caroline}, wife of one of Scrooge's debtors.
\item[\ding{100}] \textbf{Mrs Cratchit}, wife of Bob Cratchit.
\item[\ding{101}] \textbf{Belinda and Martha Cratchit}, daughters of the preceding.
\item[\ding{102}] \textbf{Mrs Dilber}, a laundress.
\item[\ding{100}] \textbf{Fan}, the sister of Scrooge.
\item[\ding{101}] \textbf{Mrs Fezziwig}, the worthy partner of Mr Fezziwig.
\end{itemize}


\pagestyle{headings}


%Some specific customizations for Christmas Carol

%store the defaults; we'll need them later
\makeatletter
\let\saved@chapterfont\scr@fnt@chapter
\let\saved@chapterprefixfont\scr@fnt@chapterprefix
\makeatother


% Make sure chapter prefix is shown
\KOMAoption{chapterprefix}{true}

% Customize the chapter prefix format
\renewcommand*{\chapterformat}{\chaptername~\thechapter}


% Update the chapter mark format to use "Stave" 
\renewcommand{\chaptername}{Stave}
\renewcommand{\thechapter}{\spelled{chapter}}

%embiggen chapter name (`Stave One') relative to chapter title (`Marley's Ghost')
\setkomafont{chapterprefix}{\chapterfont\moderatelyhuge}

%mark header with roman numeral version of chapter number
\renewcommand{\chaptermark}[1]{%
  \markboth{\chaptername\ \Roman{chapter}:\ #1}{}%
}

 %Tell ToC to list a chapter by its Roman numeral name, not the spelled-out version
\renewcommand*{\addchaptertocentry}[2]{%
  \addtocentrydefault{chapter}{}{\chaptername\ \Roman{chapter}: #2}%
}


%end of Christmas Carol customizations

\renewcommand*{\chaptermarkformat}{}
\renewcommand*{\chapterheadendvskip}{\vspace{10pt}}
\renewcommand*{\chapterheadstartvskip}{\vspace{0pt}}



\mainmatter
\flushbottom
\KOMAoptions{headings=openright}


\include{chapters/1.tex}
\include{chapters/2.tex}
\include{chapters/3.tex}
\include{chapters/4.tex}
\include{chapters/5.tex}


%reset defaults
\makeatletter
\let\scr@fnt@chapter\saved@chapterfont
\let\scr@fnt@chapterprefix\saved@chapterprefixfont
\makeatother

\cleardoublepage
\KOMAoptions{headings=openleft}
\chapter*{Colophon}
\vfill
\centering
\begin{minipage}{\textwidth}
\textit{A Christmas Carol} was first published in 1843 by Chapman \& Hall in London (UK). Illustrations by Arthur Rackham (1867\textendash1939) are taken from a version published in 1915 by J.~B. Lippincott Company in Philadelphia (USA). Grateful acknowledgment is made to the Internet Archive, which in 2008 digitised a copy of this edition belonging to the New York Public Library. Additional illustrations are from Rackham's illustrated version of Clement C. Moore's \textit{The Night Before Christmas}, published in 1915 by J.~B. Lippincott Company in Philadelphia.
\end{minipage}
\vfill
gutenberg.org/ebooks/24022
\vfill
\includegraphics[width=.5\textwidth]{bells}
\vfill
\begin{minipage}{\textwidth}
Text is set in <EB Garamond,> Georg Mayr-Duffner's free and open source implementation of Claude Garamond’s famous humanist typefaces from the mid-sixteenth century. Chapter dropcaps are set in Dieter Steffmann's <Acorn Initialen>; chapter headings are set in Dieter Steffman's <Menuetto>.
\end{minipage}
\vfill
github.com/georgd/EB-Garamond
\\steffmann.1001fonts.com
\vfill
\includegraphics[width=.5\textwidth]{bells}
\vfill
\begin{minipage}{\textwidth}
This typeset is dedicated to the public domain under a Creative Commons CC0 1.0 Universal deed.
\end{minipage}
\vfill
creativecommons.org/publicdomain/zero/1.0/
\vfill
\includegraphics[width=.5\textwidth]{bells}
\vfill
Typeset in \LaTeX{}. Last revised \today.
\thispagestyle{empty}
\end{document}